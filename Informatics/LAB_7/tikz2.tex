\begin{center}
    \noindent
    Используя TkiZ:\\
    \vspace{2mm}
\end{center}

\begin{tikzpicture}
\tikzstyle{roundnode} = [circle, draw=black, thick, minimum size=1.5cm]
\tikzstyle{edge} = [->, thick, auto]

% Создвание кругов
\node[roundnode] (b1) {b1};
\node[roundnode] (b2) [right =5cm of b1] {b2};
\node[roundnode] (b3) [right =5cm of b2] {b3};
\node[roundnode] (b4) [below =5cm of b1] {b4};
\node[roundnode] (b5) [below =5cm of b2] {b5};
\node[roundnode] (b6) [below =5cm of b3] {b6};

% Создание векторов перехода
\draw[edge]  (b1) -- node {$z_3$} (b2);
\draw[edge]  (b3) -- node [above] {$z_2$} (b2);
\draw[edge]  (b6) -- node {$z_3$} (b3);
\draw[edge]  (b2) -- node [near start] {$z_1$} (b6);
\draw[edge]  (b5) -- node [near start] {$z_3$} (b3);
\draw[edge]  (b1) -- node {$z_2$} (b5);
\draw[edge]  (b5) -- node {$z_2$} (b4);
\draw[edge]  (b4) -- node {$z_1$} (b1);

% Создание изогнутых линий
\path [->] (b1) edge [loop left] node {$z_1$}(b1); 
\path [->] (b2) edge [loop above] node {$z_2$}(b2); 
\path [->] (b4) edge [loop left] node {$z_3$}(b4); 
\path [->] (b3) edge [bend left] node [auto] {$z_1$}(b6); 
\path [->] (b6) edge [bend left] node [auto] {$z_2$}(b4); 

\end{tikzpicture}